% Options for packages loaded elsewhere
\PassOptionsToPackage{unicode}{hyperref}
\PassOptionsToPackage{hyphens}{url}
%
\documentclass[
]{article}
\usepackage{amsmath,amssymb}
\usepackage{iftex}
\ifPDFTeX
  \usepackage[T1]{fontenc}
  \usepackage[utf8]{inputenc}
  \usepackage{textcomp} % provide euro and other symbols
\else % if luatex or xetex
  \usepackage{unicode-math} % this also loads fontspec
  \defaultfontfeatures{Scale=MatchLowercase}
  \defaultfontfeatures[\rmfamily]{Ligatures=TeX,Scale=1}
\fi
\usepackage{lmodern}
\ifPDFTeX\else
  % xetex/luatex font selection
\fi
% Use upquote if available, for straight quotes in verbatim environments
\IfFileExists{upquote.sty}{\usepackage{upquote}}{}
\IfFileExists{microtype.sty}{% use microtype if available
  \usepackage[]{microtype}
  \UseMicrotypeSet[protrusion]{basicmath} % disable protrusion for tt fonts
}{}
\makeatletter
\@ifundefined{KOMAClassName}{% if non-KOMA class
  \IfFileExists{parskip.sty}{%
    \usepackage{parskip}
  }{% else
    \setlength{\parindent}{0pt}
    \setlength{\parskip}{6pt plus 2pt minus 1pt}}
}{% if KOMA class
  \KOMAoptions{parskip=half}}
\makeatother
\usepackage{xcolor}
\usepackage[margin=1in]{geometry}
\usepackage{longtable,booktabs,array}
\usepackage{calc} % for calculating minipage widths
% Correct order of tables after \paragraph or \subparagraph
\usepackage{etoolbox}
\makeatletter
\patchcmd\longtable{\par}{\if@noskipsec\mbox{}\fi\par}{}{}
\makeatother
% Allow footnotes in longtable head/foot
\IfFileExists{footnotehyper.sty}{\usepackage{footnotehyper}}{\usepackage{footnote}}
\makesavenoteenv{longtable}
\usepackage{graphicx}
\makeatletter
\def\maxwidth{\ifdim\Gin@nat@width>\linewidth\linewidth\else\Gin@nat@width\fi}
\def\maxheight{\ifdim\Gin@nat@height>\textheight\textheight\else\Gin@nat@height\fi}
\makeatother
% Scale images if necessary, so that they will not overflow the page
% margins by default, and it is still possible to overwrite the defaults
% using explicit options in \includegraphics[width, height, ...]{}
\setkeys{Gin}{width=\maxwidth,height=\maxheight,keepaspectratio}
% Set default figure placement to htbp
\makeatletter
\def\fps@figure{htbp}
\makeatother
\setlength{\emergencystretch}{3em} % prevent overfull lines
\providecommand{\tightlist}{%
  \setlength{\itemsep}{0pt}\setlength{\parskip}{0pt}}
\setcounter{secnumdepth}{5}
\ifLuaTeX
  \usepackage{selnolig}  % disable illegal ligatures
\fi
\IfFileExists{bookmark.sty}{\usepackage{bookmark}}{\usepackage{hyperref}}
\IfFileExists{xurl.sty}{\usepackage{xurl}}{} % add URL line breaks if available
\urlstyle{same}
\hypersetup{
  pdftitle={MATH 101: Calculus},
  pdfauthor={Truong-Son Van},
  hidelinks,
  pdfcreator={LaTeX via pandoc}}

\title{MATH 101: Calculus}
\author{Truong-Son Van}
\date{}

\begin{document}
\maketitle

{
\setcounter{tocdepth}{2}
\tableofcontents
}
\section*{Spring 2023}\label{spring-2023}

\section*{Syllabus}\label{syllabus}
\addcontentsline{toc}{section}{Syllabus}

\subsection*{Key info}\label{key-info}
\addcontentsline{toc}{subsection}{Key info}

\textbf{Lectures:} W-F, 11:30a-1:00p, CR 501

\textbf{Instructor:} Truong-Son Van, \textbf{Email:} \href{mailto:son.van+102@fulbright.edu.vn}{\nolinkurl{son.van+102@fulbright.edu.vn}},

\textbf{Office Hours (Instructor):} M-W: 4-5pm

\textbf{TA:} Tran Lan Phuc and Nguyen Hoang Ngoc Ha

\textbf{Office Hours (TA):}

\begin{itemize}
\item
  T-Th: 3-4pm (Phuc)
\item
  Tue: 8-10am (Ha)
\end{itemize}

\textbf{Prerequisites:} being curious.

\subsection*{Important dates}\label{important-dates}
\addcontentsline{toc}{subsection}{Important dates}

\begin{itemize}
\tightlist
\item
  Final Exam: Friday, Dec 8 2023
\item
  Midterm Exam: Wednesday, Oct 11, 2023
\item
  First day of class: Aug 28, 2022
\item
  Mid-term break: Oct 23-27
\item
  End of semester: Dec 15
\end{itemize}

\subsection*{Textbook(s) and References}\label{textbooks-and-references}
\addcontentsline{toc}{subsection}{Textbook(s) and References}

It is highly recommended that students read the textbooks.

\begin{enumerate}
\def\labelenumi{\arabic{enumi}.}
\item
  In-class worksheets.
\item
  Active Calculus by Schlicker et al.~
  (\url{https://activecalculus.org/single/book-1.html})
\item
  Thomas' Calculus: Early Transcendentals by Hass, Heil, et al.~\(14^{th}\) edition.
\item
  Calculus Early Transcendental by Stewart. \(8^{th}\) edition.
\item
  Anything you can find on Google would work.
  Calculus is a subject that people have written about
  so much. So, there's no excuse for not having access
  to the knowledge.
\end{enumerate}

\subsection*{Course description}\label{course-description}
\addcontentsline{toc}{subsection}{Course description}

How can we estimate the weight of a bridge? What price should a store set for a product so as to maximize the revenue? Calculus provides tools to answer these questions and many more. Calculus is fundamental to many scientific disciplines including physics, engineering, statistics, computer science, and economics. Using everyday language and graphs, as well as equations, data, and numerical approaches, this course will provide the essential concepts of Calculus, illustrated, and explored through a wide range of real-world examples. Students will develop their critical thinking and problem-solving skills, while also gaining a solid preparation for higher-level courses such as differential equations or statistics. The main topics are functions, limits, derivatives, and integrals.

\subsection*{Learning objectives}\label{learning-objectives}
\addcontentsline{toc}{subsection}{Learning objectives}

Two things:
- I hope this course will help you have fun and nerdy conversations with friends (or strangers on the bus), whether they're math people or not.
At the least, if you don't like the awkward silence, strike a conversation about \(\infty\)!
- I hope you will find beauty in mathematics by knowing that mathematics is all about ideas,
not computations (although computations play a big part in the usefulness of mathematics).

\subsection*{Tentative Syllabus (subject to change)}\label{tentative-syllabus-subject-to-change}
\addcontentsline{toc}{subsection}{Tentative Syllabus (subject to change)}

\subsubsection*{1. Functions and Limits}\label{functions-and-limits}
\addcontentsline{toc}{subsubsection}{1. Functions and Limits}

\begin{itemize}
\item
  Functions, graphs, and mathematical modeling
\item
  Compositions and transformations
\item
  Limits, asymptotes, and continuity
\end{itemize}

\subsubsection*{2. Derivatives}\label{derivatives}
\addcontentsline{toc}{subsubsection}{2. Derivatives}

\begin{itemize}
\item
  Slope of a straight line
\item
  Concept and definition of the derivative
\item
  Differentiability
\item
  Techniques of calculating derivatives
\item
  Shapes of curves, extrema, and optimization
\item
  Implicit differentiation
\item
  Related rates
\item
  Differentials and small changes
\item
  Antiderivatives
\end{itemize}

\subsubsection*{3. Integrals}\label{integrals}
\addcontentsline{toc}{subsubsection}{3. Integrals}

\begin{itemize}
\item
  Sequences and series
\item
  Concept and definition of the integral
\item
  The fundamental theorem of calculus
\item
  Integrals and areas
\item
  Techniques of calculating integrals
\item
  Numerical integration
\item
  Improper integrals
\end{itemize}

\subsection*{Class Policies (subject to change)}\label{class-policies-subject-to-change}
\addcontentsline{toc}{subsection}{Class Policies (subject to change)}

\subsubsection*{Lectures}\label{lectures}
\addcontentsline{toc}{subsubsection}{Lectures}

\begin{itemize}
\tightlist
\item
  If you must sleep, please don't snore. (Thanks \href{https://www.math.cmu.edu/~gautam/}{Gautam Iyer} for this amazing policy!)
\item
  Please be respectful to your classmates.
\end{itemize}

\subsubsection*{Attendance}\label{attendance}
\addcontentsline{toc}{subsubsection}{Attendance}

I don't take attendance. It's up to you to decide if it's worth it to go to
class.

\subsubsection*{Quizzes}\label{quizzes}
\addcontentsline{toc}{subsubsection}{Quizzes}

There will be quizzes every day.
There will be no make-up quizzes.
Three worst quizzes will be dropped, however.

\subsubsection*{Homework}\label{homework}
\addcontentsline{toc}{subsubsection}{Homework}

\begin{itemize}
\tightlist
\item
  Homework must be turned in \textbf{by the beginning of the class} on the due date.
\item
  All homework must be turned in via paper but you need to submit a scanned version to Canvas for proof of submission.
\item
  Collaboration for homework is strongly encouraged but you \textbf{MUST} write up your own work. Word-to-word copying is plagiarism.
\item
  Generously credit all of the people who you collaborate with at the beginning of your work.
\item
  If you use outside sources (internet, books, friends, etc.) for a particular problem, acknowledge them at the beginning of the problem.
  You will \textbf{NOT} be penalized for consulting outside sources as long as you credit them.
\item
  \textbf{Late homework policy:}

  \begin{itemize}
  \tightlist
  \item
    Late homework will NOT be accpeted. However, the two worst homeworks will be dropped.
  \end{itemize}
\item
  Advice:

  \begin{itemize}
  \tightlist
  \item
    Eat well and get enough sleep.
  \item
    Start early. One problem per day is more pleasant than seven problems in one night.
  \item
    Try to understand the materials rather than rote memorization. This will show in exams.
  \item
    Try to write clearly and demonstrate clarity of thoughts.
  \end{itemize}
\end{itemize}

\subsubsection*{Grading (subject to change)}\label{grading-subject-to-change}
\addcontentsline{toc}{subsubsection}{Grading (subject to change)}

\begin{itemize}
\tightlist
\item
  Homework: 30\%
\item
  Daily quizzes: 25\%
\item
  Midterm: 20\%
\item
  Final: 25\%
\end{itemize}

\begin{longtable}[]{@{}cc@{}}
\toprule\noalign{}
\textbf{Letter Grade} & \textbf{Percentage} \\
\midrule\noalign{}
\endhead
\bottomrule\noalign{}
\endlastfoot
A & {[}93,100{]} \\
A- & {[}90,93) \\
B+ & {[}87,90) \\
B & {[}83,87) \\
B- & {[}80, 83) \\
C+ & {[}77,80) \\
C & {[}73,77) \\
C- & {[}70,73) \\
D+ & {[}67,70) \\
D & {[}60, 66) \\
F & {[}0,60) \\
\end{longtable}

\subsection*{Time Expectations}\label{time-expectations}
\addcontentsline{toc}{subsection}{Time Expectations}

On average, you should expect to be roughly 3 hours in class per week, which are included in a total of 10 working hours per course per week. If you are finding it difficult to complete your work in time, please come talk to me ASAP so that we can diagnose the issue and adjust accordingly. If something is not working for you, please do not hesitate to raise it in one of the feedback sessions or come see me outside of class.

\subsection*{Academic Dishonesty}\label{academic-dishonesty}
\addcontentsline{toc}{subsection}{Academic Dishonesty}

As Fulbright University's Code of Academic Integrity explains: ``plagiarism occurs when a writer appropriates another's ideas, research, or writing without proper acknowledgement of the source or uses another's words without the use of quotation marks, whether intentional or not.'' All Fulbright students are responsible for familiarizing themselves with the Code of Academic Integrity.

\subsection*{Learning Support}\label{learning-support}
\addcontentsline{toc}{subsection}{Learning Support}

Please remember that ``Help is always available at FUV, if you just reach out!''. There are ample resources available to help you survive and thrive on your academic journey. The Fulbright Learning Support team can provide you with guidance in the following areas:

\begin{itemize}
\tightlist
\item
  Academic skills (e.g., Reading, Writing, Listening, Speaking and Presentation)
\item
  Study skills (e.g., Time management, Planning your Assignments, Task Management, Note-taking Skills)\\
\item
  Research-related skills (e.g., Selecting Peer-reviewed Journals, Qualitative Coding, Planning a Research Project)
\item
  Exam strategies \& Test-taking skills\\
\item
  Academic Integrity (e.g., Avoiding Plagiarism, Paraphrasing Skills, Citing and Referencing)
\item
  Individual Learning Plan (i.e., Brainstorming, Planning, Prioritizing, Monitoring, Reflection on Learning)
\item
  Making use of the Work-in-Progress Learning Guides for independently learning fundamental academic skills and study skills
\item
  Discipline-related content (e.g., Arts, History, Vietnam Studies)
\end{itemize}

Support for these areas includes Workshops, Skill Practice Sessions and Group Advising Sessions organized during the semesters, and you can also refer to the Study Skills \& Academic Skills 101 canvas module. Additionally, if you would like to have one-on-one advising/ mentoring sessions to discuss your specific academic concerns (e.g., how to improve your thesis statement, how to `polish' your academic writing style, identify your strengths and weaknesses in your academic reading skills), you can book an appointment with a learning support staff member or with a peer mentor via the booking link.\\
If you have further questions about learning support, please send an email to \href{mailto:learning.support@fulbright.edu.vn}{\nolinkurl{learning.support@fulbright.edu.vn}}

\subsection*{Wellness Center}\label{wellness-center}
\addcontentsline{toc}{subsection}{Wellness Center}

The Wellness Center support students to take care of your emotional and social health and wellbeing so you can enjoy your college experience more fully. Our offers include various wellness programs,\,free counseling service, safer community, and accessibility service for all Fulbright students.
\,
You can contact the Wellness Center via\,\href{mailto:wellness@fulbright.edu.vn}{\nolinkurl{wellness@fulbright.edu.vn}}\,or find us at the Wellness Center office on the Level 5 of our Crescent campus.
\,

\subsection*{Counseling service}\label{counseling-service}
\addcontentsline{toc}{subsection}{Counseling service}

If you are experiencing any stress or emotional concern that may be interfering with your ability to perform academically, or you want to explore more about mental health and how to live life in a more balanced way,\,you can contact the Wellness Center\,Counseling service.
Our counseling service is confidential, private, and free of charge for all Fulbright students. You can book a counseling session at this\,link\,or contact\,\href{mailto:counseling@fulbright.edu.vn}{\nolinkurl{counseling@fulbright.edu.vn}}. If you need urgent support, you can contact the International SOS via their hotline (+84 28 38298520) or access their mobile app.

\subsection*{Safe Learning Environment\,}\label{safe-learning-environment}
\addcontentsline{toc}{subsection}{Safe Learning Environment\,}

Fulbright is dedicated to a safe, supportive and non-discriminatory learning environment. Bullying, abuse, discrimination, harassment, sexual misconduct, and any other actions that create an unsafe learning environment will not be tolerated. It is the responsibility of all students to familiarize themselves with the\,Student Code of Conduct. Actions which threaten a safe campus environment - including the physical and emotional safety all students - will be investigated according to this code and may be subject to sanctions including loss of privileges, suspension, or expulsion from FUV.
\,
The\,Wellness Center offers\,Safer Community\,-\,a central point of enquiry, response, and support for concerning, threatening, or inappropriate behaviors, including sexual harassment, sexual assault, and/or any actions mentioned above. If you are feeling unsafe or unsure what to do, Safer Community will listen to you and explore options with you. Conversations are confidential unless you give your consent to involve others. You can book an appointment with Safer Community\,here\,or contact them at\,\href{mailto:safer-community@fulbright.edu.vn}{\nolinkurl{safer-community@fulbright.edu.vn}}\,for any query.

\subsection*{Accessibility Learning Service}\label{accessibility-learning-service}
\addcontentsline{toc}{subsection}{Accessibility Learning Service}

Fulbright University Vietnam commits to providing excellent student-centered services that supports diversity, inclusivity and accessibility where the student's voice and presence matters. Accessibility Learning Service provides\,support for students with conditions, including disability, long-term illness, mental health condition or being primary carers of individuals with a disability. ALS can meet with you to develop individualized learning plan, share your plan with your professors and provide continuing support if necessary. You can contact us at\,\href{mailto:wellness@fulbright.edu.vn}{\nolinkurl{wellness@fulbright.edu.vn}}\,to book an appointment. We strongly recommend that you meet with us prior to the semester start to ensure timely development and implementation of your learning plan.

\end{document}
