% Options for packages loaded elsewhere
\PassOptionsToPackage{unicode}{hyperref}
\PassOptionsToPackage{hyphens}{url}
%
\documentclass[
]{article}
\usepackage{amsmath,amssymb}
\usepackage{iftex}
\ifPDFTeX
  \usepackage[T1]{fontenc}
  \usepackage[utf8]{inputenc}
  \usepackage{textcomp} % provide euro and other symbols
\else % if luatex or xetex
  \usepackage{unicode-math} % this also loads fontspec
  \defaultfontfeatures{Scale=MatchLowercase}
  \defaultfontfeatures[\rmfamily]{Ligatures=TeX,Scale=1}
\fi
\usepackage{lmodern}
\ifPDFTeX\else
  % xetex/luatex font selection
\fi
% Use upquote if available, for straight quotes in verbatim environments
\IfFileExists{upquote.sty}{\usepackage{upquote}}{}
\IfFileExists{microtype.sty}{% use microtype if available
  \usepackage[]{microtype}
  \UseMicrotypeSet[protrusion]{basicmath} % disable protrusion for tt fonts
}{}
\makeatletter
\@ifundefined{KOMAClassName}{% if non-KOMA class
  \IfFileExists{parskip.sty}{%
    \usepackage{parskip}
  }{% else
    \setlength{\parindent}{0pt}
    \setlength{\parskip}{6pt plus 2pt minus 1pt}}
}{% if KOMA class
  \KOMAoptions{parskip=half}}
\makeatother
\usepackage{xcolor}
\usepackage[margin=1in]{geometry}
\usepackage{longtable,booktabs,array}
\usepackage{calc} % for calculating minipage widths
% Correct order of tables after \paragraph or \subparagraph
\usepackage{etoolbox}
\makeatletter
\patchcmd\longtable{\par}{\if@noskipsec\mbox{}\fi\par}{}{}
\makeatother
% Allow footnotes in longtable head/foot
\IfFileExists{footnotehyper.sty}{\usepackage{footnotehyper}}{\usepackage{footnote}}
\makesavenoteenv{longtable}
\usepackage{graphicx}
\makeatletter
\def\maxwidth{\ifdim\Gin@nat@width>\linewidth\linewidth\else\Gin@nat@width\fi}
\def\maxheight{\ifdim\Gin@nat@height>\textheight\textheight\else\Gin@nat@height\fi}
\makeatother
% Scale images if necessary, so that they will not overflow the page
% margins by default, and it is still possible to overwrite the defaults
% using explicit options in \includegraphics[width, height, ...]{}
\setkeys{Gin}{width=\maxwidth,height=\maxheight,keepaspectratio}
% Set default figure placement to htbp
\makeatletter
\def\fps@figure{htbp}
\makeatother
\setlength{\emergencystretch}{3em} % prevent overfull lines
\providecommand{\tightlist}{%
  \setlength{\itemsep}{0pt}\setlength{\parskip}{0pt}}
\setcounter{secnumdepth}{5}
\ifLuaTeX
  \usepackage{selnolig}  % disable illegal ligatures
\fi
\usepackage{bookmark}
\IfFileExists{xurl.sty}{\usepackage{xurl}}{} % add URL line breaks if available
\urlstyle{same}
\hypersetup{
  pdftitle={MATH 101: Calculus},
  pdfauthor={Truong-Son Van},
  hidelinks,
  pdfcreator={LaTeX via pandoc}}

\title{MATH 101: Calculus}
\author{Truong-Son Van}
\date{}

\begin{document}
\maketitle

{
\setcounter{tocdepth}{2}
\tableofcontents
}
\section*{Fall 2024}\label{fall-2024}

\subsection*{Key info}\label{key-info}
\addcontentsline{toc}{subsection}{Key info}

\textbf{Lectures:} M-W, 4:45pm - 6:15pm, CR1

\textbf{Instructor:} Truong-Son Van, \textbf{Email:} \href{mailto:son.van+101@fulbright.edu.vn}{\nolinkurl{son.van+101@fulbright.edu.vn}},

\textbf{Office Hours (Instructor):} Tue \& Thu: 2PM-4PM

\textbf{TA:}

\begin{itemize}
\item
  Nguyen Thien Toan (\href{mailto:toan.nguyen.220107@student.fulbright.edu.vn}{\nolinkurl{toan.nguyen.220107@student.fulbright.edu.vn}})
\item
  Tran Nam Anh (\href{mailto:anh.tran.230070@student.fulbright.edu.vn}{\nolinkurl{anh.tran.230070@student.fulbright.edu.vn}})
\end{itemize}

\textbf{Office Hours (TA):}

\begin{itemize}
\item
  Thien Toan: Fri, 8AM - 11AM
\item
  Nam Anh: Mon \& Wed 1PM - 2:30PM
\end{itemize}

\textbf{Prerequisites:} being curious.

\subsection*{Important dates}\label{important-dates}
\addcontentsline{toc}{subsection}{Important dates}

\begin{itemize}
\tightlist
\item
  Exams:

  \begin{itemize}
  \tightlist
  \item
    Mini Exam 1: Sep.~4
  \item
    Mini Exam 2: Sep.~25
  \item
    Mini Exam 3: Oct.~23
  \item
    Mini Exam 4: Nov.~20
  \item
    Final Exam: Dec.~4
  \end{itemize}
\item
  Drop dates:

  \begin{itemize}
  \tightlist
  \item
    without consequences: 4:00 PM Friday, Aug.~30
  \item
    with ``W'' on transcript: 4:00 PM Friday, Oct.~11
  \end{itemize}
\item
  Breaks:

  \begin{itemize}
  \tightlist
  \item
    First day of class:
  \item
    Independence day break: Sept 2-3, 2024
  \item
    Mid-term break: Oct 14-18
  \item
    End of semester: Dec 12
  \item
    End of semester break: Dec 16, 2024 - Jan 3, 2025
  \end{itemize}
\end{itemize}

\subsection*{Textbook(s) and References}\label{textbooks-and-references}
\addcontentsline{toc}{subsection}{Textbook(s) and References}

It is required that students read the textbooks before the class.

\begin{enumerate}
\def\labelenumi{\arabic{enumi}.}
\item
  Main Text: Calculus \href{https://openstax.org/details/books/calculus-volume-1}{Vol.1} \& \href{https://openstax.org/details/books/calculus-volume-2}{Vol. 2}, OpenStax
\item
  Calculus, Vol. 1, Tom Apostol
\item
  Active Calculus by Schlicker et al.~
  (\url{https://activecalculus.org/single/book-1.html})
\item
  Thomas' Calculus: Early Transcendentals by Hass, Heil, et al.~\(14^{th}\) edition.
\item
  Calculus Early Transcendental by Stewart. \(8^{th}\) edition.
\item
  Anything you can find on Google would work.
  Calculus is a subject that people have written about
  so much. So, there's no excuse for not having access
  to the knowledge.
\end{enumerate}

\subsection*{Course description}\label{course-description}
\addcontentsline{toc}{subsection}{Course description}

How can we estimate the weight of a bridge? What price should a store set for a product so as to maximize the revenue? Calculus provides tools to answer these questions and many more. Calculus is fundamental to many scientific disciplines including physics, engineering, statistics, computer science, and economics. Using everyday language and graphs, as well as equations, data, and numerical approaches, this course will provide the essential concepts of Calculus, illustrated, and explored through a wide range of real-world examples. Students will develop their critical thinking and problem-solving skills, while also gaining a solid preparation for higher-level courses such as differential equations or statistics. The main topics are functions, limits, derivatives, and integrals.

\subsection*{Syllabus \& Schedule}\label{syllabus-schedule}
\addcontentsline{toc}{subsection}{Syllabus \& Schedule}

\begin{itemize}
\tightlist
\item
  Week 1 (Aug.~19 - Aug.~23):

  \begin{itemize}
  \tightlist
  \item
    M: General discussion \& Review of functions. Read

    \begin{itemize}
    \tightlist
    \item
      \href{https://openstax.org/books/calculus-volume-1/pages/1-introduction}{Vol.1, Chapter 1}.
    \item
      HW: 1.1.7-21, 36-48, 1.2.75-99, 1.3.113-154, 1.4.199-215, 1.5.233-263
    \end{itemize}
  \item
    W: Preview of Calculus, Introduction to limits. Read

    \begin{itemize}
    \tightlist
    \item
      \href{https://openstax.org/books/calculus-volume-1/pages/2-1-a-preview-of-calculus}{Vol.1, 2.1}
    \item
      \href{https://openstax.org/books/calculus-volume-1/pages/2-2-the-limit-of-a-function}{Vol.1, 2.2}
    \item
      HW: 2.2.30-80
    \end{itemize}
  \end{itemize}
\item
  Week 2 (Aug.~26 - Aug.~30):

  \begin{itemize}
  \tightlist
  \item
    M: Limit Laws. Read

    \begin{itemize}
    \tightlist
    \item
      \href{https://openstax.org/books/calculus-volume-1/pages/2-3-the-limit-laws}{Vol.1, 2.3}
    \end{itemize}
  \item
    W: Continuity. Read

    \begin{itemize}
    \tightlist
    \item
      \href{https://openstax.org/books/calculus-volume-1/pages/2-4-continuity}{Vol.1, 2.4}
    \end{itemize}
  \end{itemize}
\item
  Week 3 (Sep.~2 - Sep.~6). Break and Mini Exam 1.

  \begin{itemize}
  \tightlist
  \item
    M: Break
  \item
    W: Mini Exam 1
  \end{itemize}
\item
  Week 4 (Sep.~9 - Sep.~13). Introdunction to Derivatives.

  \begin{itemize}
  \tightlist
  \item
    M: Derivative. Read

    \begin{itemize}
    \tightlist
    \item
      \href{https://openstax.org/books/calculus-volume-1/pages/3-1-defining-the-derivative}{Vol. 1, 3.1}
    \item
      \href{https://openstax.org/books/calculus-volume-1/pages/3-2-the-derivative-as-a-function}{Vol. 1, 3.2}
    \end{itemize}
  \item
    W: Derivative rules. Read

    \begin{itemize}
    \tightlist
    \item
      \href{https://openstax.org/books/calculus-volume-1/pages/3-3-differentiation-rules}{Vol. 1, 3.3}
    \item
      \href{https://openstax.org/books/calculus-volume-1/pages/3-6-the-chain-rule}{Vol. 1, 3.6}
    \end{itemize}
  \end{itemize}
\item
  Week 5 (Sep.~16 - Sep.~20). Special derivatives.

  \begin{itemize}
  \tightlist
  \item
    M: Derivatives of Trig functions, Derivatives of Inverse Functions. Read

    \begin{itemize}
    \tightlist
    \item
      \href{https://openstax.org/books/calculus-volume-1/pages/3-5-derivatives-of-trigonometric-functions}{Vol. 1, 3.5}
    \item
      \href{https://openstax.org/books/calculus-volume-1/pages/3-7-derivatives-of-inverse-functions}{Vol. 1, 3.7}
    \end{itemize}
  \item
    W: Implicit Differentiations, Derivatives of Exponential and Logarithmic Functions. Read

    \begin{itemize}
    \tightlist
    \item
      \href{https://openstax.org/books/calculus-volume-1/pages/3-8-implicit-differentiation}{Vol. 1, 3.8}
    \item
      \href{https://openstax.org/books/calculus-volume-1/pages/3-9-derivatives-of-exponential-and-logarithmic-functions}{Vol. 1, 3.9}
    \end{itemize}
  \end{itemize}
\item
  Week 6 (Sep.~23 - Sep.~27). L'Hospital Rule and Mini Exam 2.

  \begin{itemize}
  \tightlist
  \item
    M: L'Hospital's Rule. Read

    \begin{itemize}
    \tightlist
    \item
      \href{https://openstax.org/books/calculus-volume-1/pages/4-8-lhopitals-rule}{Vol. 1, 4.8}
    \end{itemize}
  \item
    W: Mini Exam 2
  \end{itemize}
\item
  Week 7 (Sep.~30 - Oct.~4). Applications of Derivatives.

  \begin{itemize}
  \tightlist
  \item
    M: Related Rates, Linear Approximations. Read

    \begin{itemize}
    \tightlist
    \item
      \href{https://openstax.org/books/calculus-volume-1/pages/4-1-related-rates}{Vol. 1, 4.1}
    \item
      \href{https://openstax.org/books/calculus-volume-1/pages/4-2-linear-approximations-and-differentials}{Vol. 1, 4.2}
    \end{itemize}
  \item
    W: Basic Optimization. Read

    \begin{itemize}
    \tightlist
    \item
      \href{https://openstax.org/books/calculus-volume-1/pages/4-3-maxima-and-minima}{Vol. 1, 4.3}
    \item
      \href{https://openstax.org/books/calculus-volume-1/pages/4-7-applied-optimization-problems}{Vol. 1, 4.7}
    \end{itemize}
  \end{itemize}
\item
  Week 8 (Oct.~7 - Oct.~11). Applications of Derivatives (cont.).

  \begin{itemize}
  \tightlist
  \item
    M: Drivatives and Shape of a Graph. Read

    \begin{itemize}
    \tightlist
    \item
      \href{https://openstax.org/books/calculus-volume-1/pages/4-5-derivatives-and-the-shape-of-a-graph}{Vol. 1, 4.5}
    \end{itemize}
  \item
    W: Newton's Method

    \begin{itemize}
    \tightlist
    \item
      \href{https://openstax.org/books/calculus-volume-1/pages/4-9-newtons-method}{Vol. 1, 4.9}
    \end{itemize}
  \end{itemize}
\item
  Week 9 (Oct.~21 - Oct.~25). Review and Mini Exam 3.

  \begin{itemize}
  \tightlist
  \item
    M: Review
  \item
    W: Mini Exam 3
  \end{itemize}
\item
  Week 10 (Oct.~28 - Nov.~1). Antiderivative and Integration.

  \begin{itemize}
  \tightlist
  \item
    M: Antiderivatives. Read

    \begin{itemize}
    \tightlist
    \item
      \href{https://openstax.org/books/calculus-volume-1/pages/4-10-antiderivatives}{Vol. 1, 4.10}
    \end{itemize}
  \item
    W: Introduction to Integration. Read

    \begin{itemize}
    \tightlist
    \item
      \href{https://openstax.org/books/calculus-volume-1/pages/5-1-approximating-areas}{Vol. 1, 5.1}
    \end{itemize}
  \end{itemize}
\item
  Week 11 (Nov.~4 - Nov.~8)

  \begin{itemize}
  \tightlist
  \item
    M: Definite Integral. Read

    \begin{itemize}
    \tightlist
    \item
      \href{https://openstax.org/books/calculus-volume-1/pages/5-2-the-definite-integral}{Vol. 1, 5.2}
    \end{itemize}
  \item
    W: Fundamental Theorem of Calculus. Read

    \begin{itemize}
    \tightlist
    \item
      \href{https://openstax.org/books/calculus-volume-1/pages/5-3-the-fundamental-theorem-of-calculus}{Vol. 1, 5.3}
    \end{itemize}
  \end{itemize}
\item
  Week 12 (Nov.~11 - Nov.~15)

  \begin{itemize}
  \tightlist
  \item
    M: Integration Formulas and the Net Change Theorem. Read

    \begin{itemize}
    \tightlist
    \item
      \href{https://openstax.org/books/calculus-volume-1/pages/5-4-integration-formulas-and-the-net-change-theorem}{Vol. 1, 5.4}
    \end{itemize}
  \item
    W: Substitution. Read

    \begin{itemize}
    \tightlist
    \item
      \href{https://openstax.org/books/calculus-volume-1/pages/5-5-substitution}{Vol. 1, 5.5}
    \end{itemize}
  \end{itemize}
\item
  Week 13 (Nov.~18 - Nov.~22)

  \begin{itemize}
  \tightlist
  \item
    M: Areas between Curves. Read

    \begin{itemize}
    \tightlist
    \item
      \href{https://openstax.org/books/calculus-volume-1/pages/6-1-areas-between-curves}{Vol. 1, 6.1}
    \end{itemize}
  \item
    W: Mini Exam 4
  \end{itemize}
\item
  Week 14 (Nov.~25 - Nov.~29)

  \begin{itemize}
  \tightlist
  \item
    M: Determining Volumes by Slicing and Volumes of Revolution: Cylindrical Shells. Read

    \begin{itemize}
    \tightlist
    \item
      \href{https://openstax.org/books/calculus-volume-1/pages/6-3-volumes-of-revolution-cylindrical-shells}{Vol. 1, 6.3}
    \item
      \href{https://openstax.org/books/calculus-volume-1/pages/6-2-determining-volumes-by-slicing}{Vol. 1, 6.2}
    \end{itemize}
  \item
    W: Exponential Growth and Decay. Read

    \begin{itemize}
    \tightlist
    \item
      \href{https://openstax.org/books/calculus-volume-1/pages/6-8-exponential-growth-and-decay}{Vol. 1, 6.8}
    \end{itemize}
  \end{itemize}
\item
  Week 15 (Dec.~2 - Dec.~6)

  \begin{itemize}
  \tightlist
  \item
    M: Review
  \item
    W: Final
  \end{itemize}
\end{itemize}

\subsection*{Class Policies (subject to change)}\label{class-policies-subject-to-change}
\addcontentsline{toc}{subsection}{Class Policies (subject to change)}

\subsubsection*{Lectures}\label{lectures}
\addcontentsline{toc}{subsubsection}{Lectures}

\begin{itemize}
\tightlist
\item
  If you must sleep, please don't snore. (Thanks \href{https://www.math.cmu.edu/~gautam/}{Gautam Iyer} for this amazing policy!)
\item
  Please be respectful to your classmates.
\end{itemize}

\subsubsection*{Attendance}\label{attendance}
\addcontentsline{toc}{subsubsection}{Attendance}

I don't take attendance. It's up to you to decide if it's worth it to go to
class.

\subsubsection*{Homework}\label{homework}
\addcontentsline{toc}{subsubsection}{Homework}

Homework is highly recommended and often required to learn any Math
subject. However, I will not collect your homework.
As adults, you are responsibile for your own learning.

Instead of spending time grading your homework, I decide to
increase my office hour time so you have more access to me,
should you need it.

\subsubsection*{Grading (subject to change)}\label{grading-subject-to-change}
\addcontentsline{toc}{subsubsection}{Grading (subject to change)}

\begin{itemize}
\tightlist
\item
  Mini Exams: 15\% x 4
\item
  Final: 40\%
\end{itemize}

\begin{longtable}[]{@{}cc@{}}
\toprule\noalign{}
\textbf{Letter Grade} & \textbf{Percentage} \\
\midrule\noalign{}
\endhead
\bottomrule\noalign{}
\endlastfoot
A & {[}93,100{]} \\
A- & {[}90,93) \\
B+ & {[}87,90) \\
B & {[}83,87) \\
B- & {[}80, 83) \\
C+ & {[}77,80) \\
C & {[}73,77) \\
C- & {[}70,73) \\
D+ & {[}67,70) \\
D & {[}60, 66) \\
F & {[}0,60) \\
\end{longtable}

\subsection*{Time Expectations}\label{time-expectations}
\addcontentsline{toc}{subsection}{Time Expectations}

On average, you should expect to be roughly 3 hours in class per week, which are included in a total of 10 working hours per course per week. If you are finding it difficult to complete your work in time, please come talk to me ASAP so that we can diagnose the issue and adjust accordingly. If something is not working for you, please do not hesitate to raise it in one of the feedback sessions or come see me outside of class.

\subsection*{Academic Dishonesty}\label{academic-dishonesty}
\addcontentsline{toc}{subsection}{Academic Dishonesty}

As Fulbright University's Code of Academic Integrity explains: ``plagiarism occurs when a writer appropriates another's ideas, research, or writing without proper acknowledgement of the source or uses another's words without the use of quotation marks, whether intentional or not.'' All Fulbright students are responsible for familiarizing themselves with the Code of Academic Integrity.

\subsection*{Learning Support}\label{learning-support}
\addcontentsline{toc}{subsection}{Learning Support}

Please remember that ``Help is always available at FUV, if you just reach out!''. There are ample resources available to help you survive and thrive on your academic journey. The Fulbright Learning Support team can provide you with guidance in the following areas:

\begin{itemize}
\tightlist
\item
  Academic skills (e.g., Reading, Writing, Listening, Speaking and Presentation)
\item
  Study skills (e.g., Time management, Planning your Assignments, Task Management, Note-taking Skills)\\
\item
  Research-related skills (e.g., Selecting Peer-reviewed Journals, Qualitative Coding, Planning a Research Project)
\item
  Exam strategies \& Test-taking skills\\
\item
  Academic Integrity (e.g., Avoiding Plagiarism, Paraphrasing Skills, Citing and Referencing)
\item
  Individual Learning Plan (i.e., Brainstorming, Planning, Prioritizing, Monitoring, Reflection on Learning)
\item
  Making use of the Work-in-Progress Learning Guides for independently learning fundamental academic skills and study skills
\item
  Discipline-related content (e.g., Arts, History, Vietnam Studies)
\end{itemize}

Support for these areas includes Workshops, Skill Practice Sessions and Group Advising Sessions organized during the semesters, and you can also refer to the Study Skills \& Academic Skills 101 canvas module. Additionally, if you would like to have one-on-one advising/ mentoring sessions to discuss your specific academic concerns (e.g., how to improve your thesis statement, how to `polish' your academic writing style, identify your strengths and weaknesses in your academic reading skills), you can book an appointment with a learning support staff member or with a peer mentor via the booking link.\\
If you have further questions about learning support, please send an email to \href{mailto:learning.support@fulbright.edu.vn}{\nolinkurl{learning.support@fulbright.edu.vn}}

\subsection*{Wellness Center}\label{wellness-center}
\addcontentsline{toc}{subsection}{Wellness Center}

The Wellness Center support students to take care of your emotional and social health and wellbeing so you can enjoy your college experience more fully. Our offers include various wellness programs,\,free counseling service, safer community, and accessibility service for all Fulbright students.
\,
You can contact the Wellness Center via\,\href{mailto:wellness@fulbright.edu.vn}{\nolinkurl{wellness@fulbright.edu.vn}}\,or find us at the Wellness Center office on the Level 5 of our Crescent campus.
\,

\subsection*{Counseling service}\label{counseling-service}
\addcontentsline{toc}{subsection}{Counseling service}

If you are experiencing any stress or emotional concern that may be interfering with your ability to perform academically, or you want to explore more about mental health and how to live life in a more balanced way,\,you can contact the Wellness Center\,Counseling service.
Our counseling service is confidential, private, and free of charge for all Fulbright students. You can book a counseling session at this\,link\,or contact\,\href{mailto:counseling@fulbright.edu.vn}{\nolinkurl{counseling@fulbright.edu.vn}}. If you need urgent support, you can contact the International SOS via their hotline (+84 28 38298520) or access their mobile app.

\subsection*{Safe Learning Environment\,}\label{safe-learning-environment}
\addcontentsline{toc}{subsection}{Safe Learning Environment\,}

Fulbright is dedicated to a safe, supportive and non-discriminatory learning environment. Bullying, abuse, discrimination, harassment, sexual misconduct, and any other actions that create an unsafe learning environment will not be tolerated. It is the responsibility of all students to familiarize themselves with the\,Student Code of Conduct. Actions which threaten a safe campus environment - including the physical and emotional safety all students - will be investigated according to this code and may be subject to sanctions including loss of privileges, suspension, or expulsion from FUV.
\,
The\,Wellness Center offers\,Safer Community\,-\,a central point of enquiry, response, and support for concerning, threatening, or inappropriate behaviors, including sexual harassment, sexual assault, and/or any actions mentioned above. If you are feeling unsafe or unsure what to do, Safer Community will listen to you and explore options with you. Conversations are confidential unless you give your consent to involve others. You can book an appointment with Safer Community\,here\,or contact them at\,\href{mailto:safer-community@fulbright.edu.vn}{\nolinkurl{safer-community@fulbright.edu.vn}}\,for any query.

\subsection*{Accessibility Learning Service}\label{accessibility-learning-service}
\addcontentsline{toc}{subsection}{Accessibility Learning Service}

Fulbright University Vietnam commits to providing excellent student-centered services that supports diversity, inclusivity and accessibility where the student's voice and presence matters. Accessibility Learning Service provides\,support for students with conditions, including disability, long-term illness, mental health condition or being primary carers of individuals with a disability. ALS can meet with you to develop individualized learning plan, share your plan with your professors and provide continuing support if necessary. You can contact us at\,\href{mailto:wellness@fulbright.edu.vn}{\nolinkurl{wellness@fulbright.edu.vn}}\,to book an appointment. We strongly recommend that you meet with us prior to the semester start to ensure timely development and implementation of your learning plan.

\end{document}
