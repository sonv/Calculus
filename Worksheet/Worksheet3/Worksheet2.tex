\documentclass[12pt]{amsart}
\usepackage{amsaddr}
\usepackage[draft]{../marktext} 
%% Remove draft for real article, put twocolumn for two columns
\usepackage[draft]{../svmacro}
\usepackage[utf8]{inputenc}
\usepackage[style=alphabetic, backend=biber]{biblatex}
\addbibresource{bibliography.bib}

%% commentary bubble
\newcommand{\SV}[2][]{\sidenote[colback=green!10]{\textbf{SV\xspace #1:} #2}}

%% Title 
\title{ Worksheet 3 }
\author{MATH 101}
\address{Fulbright University, Ho Chi Minh City, Vietnam}

%\author{Co-author}
%\address{  }
%\email {  }
%
\date{\today}

\begin{document}

\maketitle
\begin{theorem}[Squeeze Theorem]
	Let \( f(x) \), \( g(x) \), and \( h(x) \) be functions defined for all \( x \neq a \) over an open interval containing \( a \). Suppose:

	\[
		f(x) \leq g(x) \leq h(x) \quad \text{for all } x \neq a \text{ in an open interval containing } a
	\]

	and

	\[
		\lim_{x \to a} f(x) = L = \lim_{x \to a} h(x)
	\]

	where \( L \) is a real number. Then,

	\[
		\lim_{x \to a} g(x) = L.
	\]
\end{theorem}

\begin{definition}
	Let \( f(x) \) be a function. If any of the following conditions hold, then the line \( x = a \) is a \textbf{vertical} asymptote of \( f(x) \).

	\[
		\lim_{x \to a^-} f(x) = +\infty \text{ or } -\infty
	\]
	\[
		\lim_{x \to a^+} f(x) = +\infty \text{ or } -\infty
	\]
	\[
		\text{or}
	\]
	\[
		\lim_{x \to a} f(x) = +\infty \text{ or } -\infty
	\]
\end{definition}
\printbibliography
%\bibliography{refs}
%\bibliographystyle{halpha-abbrv}


\end{document}
