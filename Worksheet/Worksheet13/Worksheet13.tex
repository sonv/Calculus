\documentclass[12pt]{amsart}
\usepackage{amsaddr}
\usepackage[draft]{../marktext} 
%% Remove draft for real article, put twocolumn for two columns
\usepackage[draft]{../svmacro}
\usepackage[utf8]{inputenc}
\usepackage[style=alphabetic, backend=biber]{biblatex}
\addbibresource{bibliography.bib}

%% commentary bubble
\newcommand{\SV}[2][]{\sidenote[colback=green!10]{\textbf{SV\xspace #1:} #2}}

%% Title 
\title{ Worksheet 13}
\author{MATH 101}
\address{Fulbright University, Ho Chi Minh City, Vietnam}

%\author{Co-author}
%\address{  }
%\email {  }
%
\date{\today}

\begin{document}

\maketitle



\section*{Optimization}


\begin{theorem}[First derivative test]
	Suppose that $f$ is a continuous function over an interval $I$ containing a critical point \( c \). If \( f \) is differentiable over \( I \), except possibly at point \( c \), then \( f(c) \) satisfies one of the following descriptions:

	\begin{enumerate}
		\item If \( f' \) changes sign from positive when \( x < c \) to negative when \( x > c \), then \( f(c) \) is a local maximum of \( f \).
		\item If \( f' \) changes sign from negative when \( x < c \) to positive when \( x > c \), then \( f(c) \) is a local minimum of \( f \).
		\item If \( f' \) has the same sign for \( x < c \) and \( x > c \), then \( f(c) \) is neither a local maximum nor a local minimum of \( f \).
	\end{enumerate}
\end{theorem}

\begin{theorem}[Closed interval method]
	To find the absolute extrema of a continuous function $f$ on a closed interval $[a,b]$ we follow the
	following steps:
	\begin{enumerate}
		\item Find the critical points and the values of $f$ at those points
		\item Find the values of $f$ at the end points
		\item Compare all the values from the above steps to find absolute max/min
	\end{enumerate}

\end{theorem}

\begin{definition}
	Let $f$ be a function that is differentiable over an interval $I$.
	If $f'$ is increasing over $I$, we say $f$ is concave up.
	If $f'$ is decreasing over $I$, we say $f$ is concave down.
	The point where $f$ changes concavity is called inflection point.
\end{definition}

\begin{theorem}
	Let $f$ be a function that is twice differentiable over an interval $I$.
	\begin{enumerate}
		\item If $f''(x) > 0$, then $f$ is concave up.
		\item If $f''(x) < 0$, then $f$ is concave down.
	\end{enumerate}
\end{theorem}


\begin{theorem}[Second derivative test]
	Suppose $f'(c) = 0$ and $f''$ is continuous over an interval $I$ containing $c$.
	\begin{enumerate}
		\item If $f''(c)>0$, then $f$ has a local minimum at $c$.
		\item If $f''(c)<0$, then $f$ has a local maximum at $c$.
		\item If $f''(c)=0$, the test is inconclusive.
	\end{enumerate}
\end{theorem}

\end{document}
