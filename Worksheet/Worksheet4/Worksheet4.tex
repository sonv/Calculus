\documentclass[12pt]{amsart}
\usepackage{amsaddr}
\usepackage[draft]{../marktext} 
%% Remove draft for real article, put twocolumn for two columns
\usepackage[draft]{../svmacro}
\usepackage[utf8]{inputenc}
\usepackage[style=alphabetic, backend=biber]{biblatex}
\addbibresource{bibliography.bib}

%% commentary bubble
\newcommand{\SV}[2][]{\sidenote[colback=green!10]{\textbf{SV\xspace #1:} #2}}

%% Title 
\title{ Worksheet 4}
\author{MATH 101}
\address{Fulbright University, Ho Chi Minh City, Vietnam}

%\author{Co-author}
%\address{  }
%\email {  }
%
\date{\today}

\begin{document}

\maketitle

Please make sure you have a graphical example for each of the definitions below.

\begin{theorem}[Composite Function Theorem]
	If $f(x)$ is continuous at $L$ and $\lim_{x \to a} g(x) = L$, then
	\begin{equation*}
		\lim_{x\to a} f(g(x)) = f(\lim_{x\to a} g(x)) = f(L) \,.
	\end{equation*}
\end{theorem}

\begin{question}
	Find the limits:
	\begin{enumerate}
		\item $\lim_{x \to \pi } \sin(x + \sin x)$
		      \vspace{5cm}
		\item $\displaystyle \lim_{x \to 1} \ln\left( \frac{5 - x^2}{1 + x} \right)$
		      \vspace{5cm}
	\end{enumerate}

\end{question}



\begin{theorem}[Intermediate Value Theorem]
	Let $f$ be continuous over a closed, bounded interval $[a,b]$.
	If $z$ is any real number between $f(a)$ and $f(b)$, then there exists
	a number $c$ in $[a,b]$ so that $f(c) = z$.
\end{theorem}


\begin{question}
	\begin{enumerate}
		\item Show that there is a solution to the equation
		      $$x^3 + 5x^2 + 3x - 9 = 0$$
		      \vspace{5cm}
		\item Show that $f(x) = x^3 - x^2 -3x + 1$ has at least one zero over $[0,1]$.
		      \vspace{5cm}
		\item Show that $f(x) = x - \cos x$ has at least one zero.
	\end{enumerate}
\end{question}


\begin{question}
	Do the graphs of $f(x) = 2^x$ and $g(x) = x^3$ cross at some point?
\end{question}


\newpage


\begin{question}
	A ball is thrown into the air and the vertical position is given by
	$$ x(t) = -4.9t^2 + 25t + 5 .$$
	Use the Intermediate Value Theorem to show that the ball must land on the ground sometime between 5 sec and 6 sec after the throw.
\end{question}


\vspace{5cm}

\begin{question}
	A Tibetan monk leaves the monastery at 7:00 am and
	takes his usual path to the top of the mountain, arriving at
	7:00 pm. The following morning, he starts at 7:00 am at the
	top and takes the same path back, arriving at the monastery
	at 7:00 pm. Use the Intermediate Value Theorem to show
	that there is a point on the path that the monk will cross at
	exactly the same time of day on both days.
\end{question}

\printbibliography


\end{document}
